\chapter{Исследовательский раздел}

\section{Технические характеристики}

Исследование проводилось на ЭВМ со следующими характеристиками:
\begin{itemize}[label=--]
    \item операционная система Ubuntu 22.04.5 LTS;
    \item объем оперативной памяти 16 ГБ;
    \item процессор Intel Core i7-8700K CPU 3.70Ггц × 12~\cite{processor}.
\end{itemize}

\section{Проведение исследования}

Было проведено исследование зависимости времени работы алгоритмов умножения матриц от размера входных квадратных матриц. Для каждого размера было проведено $10$ замеров времени. Измерение времени проводилось с помощью функции \textit{process\_time} из модуля \textit{time}.

В таблице~\ref{tab:comparison} приведены результаты замера процессорного времени работы алгоритмов.

\begin{table}
\centering
\caption{Зависимость времени работы алгоритмов от линейного размера входных матриц}
\begin{tabular}{|m{3cm}|c|c|m{5cm}|}
\hline
\textbf{Размер матрицы} & \textbf{Стандартный, с} & \textbf{Виноград, c} & \textbf{Оптимизированный Виноград, c} \\ \hline
21  & 0.0013 & 0.0014 & 0.0013 \\ \hline
41  & 0.0077 & 0.0100 & 0.0090 \\ \hline
61  & 0.0248 & 0.0318 & 0.0283 \\ \hline
81  & 0.0579 & 0.0746 & 0.0660 \\ \hline
101 & 0.1118 & 0.1436 & 0.1268 \\ \hline
121 & 0.1924 & 0.2462 & 0.2174 \\ \hline
141 & 0.3040 & 0.3882 & 0.3433 \\ \hline
161 & 0.4522 & 0.5765 & 0.5086 \\ \hline
181 & 0.6419 & 0.8139 & 0.7194 \\ \hline
201 & 0.8778 & 1.1183 & 0.9856 \\ \hline
221 & 1.1668 & 1.4779 & 1.3059 \\ \hline
241 & 1.5149 & 1.9216 & 1.6916 \\ \hline
261 & 1.9474 & 2.4351 & 2.1496 \\ \hline
281 & 2.4148 & 3.0461 & 2.7247 \\ \hline
301 & 2.9777 & 3.7617 & 3.3824 \\ \hline
321 & 3.6667 & 4.5926 & 4.1631 \\ \hline
341 & 4.3503 & 5.5051 & 5.0158 \\ \hline
\end{tabular}
\label{tab:comparison}
\end{table}

\clearpage

На рисунке~\ref{img:plot.png} продемонстрированы графики, построенные на основе табличных данных.

\includeimage
{plot.png}
{f}
{h}
{1 \textwidth} 
{Зависимость времени работы алгоритмов от линейного размера входных матриц} 

\section*{Вывод}

Исследование процессорного времени подтвердило теоретические расчёты трудоёмкости алгоритмов. Оптимизированный алгоритм Винограда для размеров матриц от 21 до 341 в среднем требует на 12.17\% процессорного времени меньше, чем неоптимизированный алгоритм. Но выполненных оптимизаций оказалось не достаточно для достижения показателей, меньших чем у стандартного алгоритма умножения матриц.

