\chapter{Технологический раздел}

В данном разделе будет представлена реализация алгоритмов поиска редакционного расстояния. Также будут указаны средства реализации алгоритмов и результаты тестирования.

\section{Средства реализации}

Для реализации был выбран язык программирования Python~\cite{python}. Выбор обусловлен наличием функции вычисления процессорного времени time.process\_time() в библиотеке time~\cite{time}.

\section{Реализация алгоритмов}

В листинге~\ref{lst:matrix.txt} представлена функция для создания матрицы размером $rows \times cols$.

\includelisting
{matrix.txt}
{Функция создания матрицы}

В листинге~\ref{lst:standard.txt} продемонстрирована реализация стандартного алгоритма умножения матриц.

\clearpage

\includelisting
{standard.txt}
{Реализация стандартного алгоритма умножения матриц}

\clearpage

В листингах~\ref{lst:winograd.txt}~и~\ref{lst:winograd_opt.txt}  представлены реализации алгоритмов Винограда и оптимизированного алгоритма Винограда.

\includelisting
{winograd.txt}
{Реализация алгоритма Винограда}

\clearpage

\includelisting
{winograd_opt.txt}
{Реализация оптимизированного алгоритма Винограда}

\clearpage

В листинге~\ref{lst:arr.txt} изображены функции для инициализации вспомогательных массивов для оптимизированного алгоритма Винограда.

\includelisting
{arr.txt}
{Реализация оптимизированного алгоритма Винограда}

\section{Тестирование}

В таблице~\ref{table:matrix_tests} представлены тесты для алгоритмов умножения матриц, входные данные указаны в~(\ref{eq:matrix_definitions}). Тестирование проводилось по методологии чёрного ящика. Все тесты пройдены успешно.

\begin{align}
&m_1 = [] \\
&m_2 = \begin{bmatrix} 
1 & 2 & 3 \\ 
4 & 5 & 6 
\end{bmatrix} \notag \\
&m_3 = \begin{bmatrix} 
7 & 8 \\ 
9 & 10 \\ 
11 & 12 
\end{bmatrix} \notag \\
&m_4 = \begin{bmatrix} 
58 & 64 \\ 
139 & 154 
\end{bmatrix} \notag \\
&m_5 = \begin{bmatrix} 
5 & 6 & 7 
\end{bmatrix} \label{eq:matrix_definitions}
\end{align}

\begin{table}[htb]
\caption{Тесты для алгоритмов умножения матриц}
\small
\centering
\begin{tabular}{|c|p{3cm}|p{4cm}|p{2cm}|p{2.1cm}|p{2.4cm}|}
    \hline
    № & \textbf{Описание теста} & \textbf{Алгоритм} & \textbf{Входные данные} & \textbf{Выходные данные} & \textbf{Ожидаемые данные} \\ \hline
    1 & Пустые матрицы & Стандартный & $m_1, m_1$ & None & None \\ \hline
    2 & Пустые матрицы & Виноград & $m_1, m_1$ & None & None \\ \hline
    3 & Пустые матрицы & Оптимизированный Виноград & $m_1, m_1$ & None & None \\ \hline
    4 & Несовместимые матрицы & Стандартный & $m_2, m_5$ & None & None \\ \hline
    5 & Несовместимые матрицы & Виноград & $m_2, m_5$ & None & None \\ \hline
    6 & Несовместимые матрицы & Оптимизированный Виноград & $m_2, m_5$ & None & None \\ \hline
    7 & Корректные данные & Стандартный & $m_2, m_3$ & $m_4$ & $m_4$ \\ \hline
    8 & Корректные данные & Виноград & $m_2, m_3$ & $m_4$ & $m_4$ \\ \hline
    9 & Корректные данные & Оптимизированный Виноград & $m_2, m_3$ & $m_4$ & $m_4$ \\ \hline
\end{tabular}
\label{table:matrix_tests}
\end{table}


