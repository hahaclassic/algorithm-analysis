\chapter*{ЗАКЛЮЧЕНИЕ}
\addcontentsline{toc}{chapter}{ЗАКЛЮЧЕНИЕ}

Матричные алгоритмы Левенштейна и Дамерау~–--~Левенштейна имеют схожее время работы и потребление памяти, поскольку последний использует лишь одну дополнительную переменную для учёта транспозиций. Рекурсивный алгоритм Левенштейна обладает худшим времем работы, но использование мемоизации улучшает его производительность --- для строк длиной 6 символов скорость работы увеличивается в 30 раз. Однако мемоизация требует дополнительных затрат памяти для хранения промежуточных результатов.

Рекурсивные реализации ограничены глубиной стека, что делает их непригодными для работы с длинными строками. Выбор между алгоритмами Левенштейна и Дамерау~---~Левенштейна зависит от необходимости учёта операции транспозиции.

Все задачи лабораторной работы выполнены: 

\begin{itemize}[label=--]
	\item построены схемы рекурсивных и нерекурсивных алгоритмов поиска расстояния Левенштейна и Дамерау~---~Левенштейна;
	\item выполнена программная реализация перечисленных выше алгоритмов;
	\item проведён анализ процессорного времени работы и потребляемой памяти для реализаций алгоритмов;
        \item полученные результаты описаны и обоснованы.
\end{itemize}

Цель лабораторной работы достигнута: было проведено исследование различных реализаций алгоритмов нахождения расстояний Левенштейна и Дамерау~---~Левенштейна на основе разработанного ПО. 
