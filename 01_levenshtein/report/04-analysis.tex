\section{Аналитическая часть}

В этом разделе будут рассмотрены алгоритмы нахождения расстояний Левенштейна и Дамерау~---~Левенштейна. 
 
За редакционное расстояние принимают минимальное количество операций над строкой, с помощью которых она может быть преобразована в другую. Для каждой операции над строкой вводится условная цена. Суммарная стоимость всех произведённых операций и будет являться редакционным расстоянием между данными строками. Определим стоимости каждой редакционной операции:

\begin{enumerate}
	\item $\omega(\alpha,\beta)~=~1$~---~стоимость замены при $\alpha \neq \beta$;
	\item $\omega(\alpha,\alpha)~=~0$~---~стоимость эквивалентной замены;
	\item $\omega(\varnothing,\beta)~=~1$~---~стоимость вставки;
	\item $\omega(\alpha,\varnothing)~=~1$~---~стоимость удаления.
\end{enumerate}

Пусть исходные строки \(S_1\) и \(S_2\) заданы посимвольно массивами \(S_1[1...n]\) и \(S_2[1...m]\). Пусть \(D(i, j)\) --- функция, значением которой является редакционное расстояние между подстроками \(S_1[1...i]\) и \(S_2[1...j]\). Функция \(D(i, j)\) определяет минимальное число редакционных операций для преобразования первых \(i\) символов строки \(S_1\) в первые \(j\) символов строки \(S_2\). Таким образом, редакционное расстояние между строками \(S_1\) и \(S_2\) в точности равно  \(D(n, m)\). Для описания алгоритма вычисления расстояния Левенштейна используется рекуррентное соотношение

\begin{equation}
\label{eq:lev_formula}
    D(i, j) = \begin{cases}
        0, & i = 0, j = 0\\
        i, & j = 0, i > 0\\
        j, & i = 0, j > 0\\
        D(i-1, j-1), & i > 0, j > 0, S_1[i] = S_2[j]\\
        1 + \min (\\
            \qquad D(i, j-1),\\
            \qquad D(i-1, j), & i > 0, j > 0, S_1[i] \neq S_2[j]\\
            \qquad D(i-1, j-1)\\
            ).\\
    \end{cases}
\end{equation}

Так как для вычисления расстояния Дамерау~---~Левенштейна вводится дополнительная операция транспозиции, соотношение~\ref{eq:lev_formula} примет вид:

\begin{equation}
\label{eqn:dam_lev_formula}
	D(i, j) = 
	\begin{cases}
		0, & i = 0, j = 0\\
            i, & j = 0, i > 0\\
            j, & i = 0, j > 0\\
            D(i-1, j-1), & i > 0, j > 0, S_1[i] = S_2[j]\\
		1 + min \begin{cases}
			D(i, j - 1),\\
			D(i - 1, j),\\
			D(i - 1, j - 1), \\
			D(i - 2, j - 2), \\
		\end{cases}
		& \begin{aligned}
			& \text{если} \ i > 1, j > 1, \\
			& S_{1}[i] = S_{2}[j - 1], \\
			& S_{1}[i - 1] = S_{2}[j], \\
		\end{aligned}\\
		1 + min \begin{cases}
			D(i, j - 1),\\
			D(i - 1, j), \\
			D(i - 1, j - 1), \\
		\end{cases}
		 & \text{иначе.}
	\end{cases}
\end{equation}

\subsection{Рекурсивные алгоритмы}\label{sect:recur_lev}

Простой рекурсивный алгоритм нахождения расстояния Левенштейна реализует рекуррентное соотношение~\ref{eq:lev_formula} на прямую. Но данный рекурсивный подход <<сверху-вниз>> (от англ. \textit{top-down}) к вычислению \(D(n, m)\) крайне неэффективен при больших значениях \(n\) и \(m\). Проблема в том, что количество рекурсивных вызовов растёт экспоненциально с увеличением \(n\) и \(m\). Но существует только \((n + 1) \times (m + 1)\) комбинаций \(i\) и \(j\), поэтому возможно только \((n + 1) \times (m + 1)\) различных рекурсивных вызовов. Следовательно, неэффективность подхода <<сверху-вниз>> обусловлена огромным количеством избыточных рекурсивных вызовов процедуры~\cite{cambridge}.

Для того, чтобы исключить повторные вычисления, в рекурсивном алгоритме нахождения расстояния Левенштейна с кэшем используется мемоизация --- способ оптимизации, смысл которого заключается в сохранении результата предыдущих вызовов функции \(D(i, j)\) для некоторых \(i\) и \(j\). Тогда при каждом вызове \(D(i, j)\) будет проверяться, вызывалась ли функция для заданных \(i\) и \(j\):

\begin{itemize}
    \item если не вызывалась, то функция \(D(i, j)\) вызывается, и результат её выполнения сохраняется;
    \item если вызывалась, то используется сохранённый результат.
\end{itemize}

\subsection{Матричные алгоритмы}\label{sect:dyn_lev}

Другой вариант нахождения расстояния Левенштейна --- использование подхода <<снизу-вверх>> (от англ. \textit{bottom-up}), который делится на два этапа:

\begin{enumerate}
    \item вычисление \(D(i, j)\) для наименьших возможных значений \(i\) и \(j\);
    \item вычисление \(D(i, j)\) для остальных возрастающих значений \(i\) и \(j\)~\cite{cambridge}.
\end{enumerate}

Обычно такой подход реализован с помощью матрицы результатов \(M\) размером \((n + 1) \times (m + 1)\), которая содержит значения \(D(i, j)\) для всех пар \(i\)~и~\(j\).

\begin{table}[htb]
\caption{\centering Матрица результатов, которая будет использоваться для расчёта редакционного расстояния между словами \textit{vintner} и \textit{writers}.}
\small
\centering\begin{tabular}{|c|c|c|c|c|c|c|c|c|c|}
    \hline
    \multirow{2}{*}{$D(i,j)$} & \multirow{2}{*}{} & \multirow{1}{*}{} & \multirow{1}{*}{\textbf{w}} & \multirow{1}{*}{\textbf{r}} & \multirow{1}{*}{\textbf{i}} & \multirow{1}{*}{\textbf{t}} & \multirow{1}{*}{\textbf{e}} & \multirow{1}{*}{\textbf{r}} & \multirow{1}{*}{\textbf{s}} \\ \cline{3-10}
    & & 0 & 1 & 2 & 3 & 4 & 5 & 6 & 7 \\ \hline
    & 0 & 0 & 1 & 2 & 3 & 4 & 5 & 6 & 7 \\ \hline
    \textbf{v} & 1 & 1 & & & & & & & \\ \hline
    \textbf{i} & 2 & 2 & & & & & & & \\ \hline
    \textbf{n} & 3 & 3 & & & & & & & \\ \hline
    \textbf{t} & 4 & 4 & & & & & & & \\ \hline
    \textbf{n} & 5 & 5 & & & & & & & \\ \hline
    \textbf{e} & 6 & 6 & & & & & & & \\ \hline
    \textbf{r} & 7 & 7 & & & & & & & \\ \hline
\end{tabular}
\label{table:matrix}
\end{table}

Значение в ячейке \(M[i][j]\) соответствует значению \(D(i, j)\). При вычислении значения для конкретной ячейки используются только ячейки \(M(i - 1, j - 1)\), \(M(i, j - 1)\) и \(M(i - 1, j)\), а также два символа \(S_1(i)\) и \(S_2(j)\). Таблица заполняется в соответствии с рекуррентным соотношением~\ref{eq:lev_formula}.

Стоит отметить, что для вычисления следующего элемента \(D(i, j)\) нет необходимости хранить всю матрицу значений, т.~к. используются только текущая и предыдущая строки. Поэтому для оптимизации использования памяти можно хранить только 2 строки.

Для матричного алгоритма нахождения расстояния Дамерау-Левенштейна все рассуждения аналогичны, но используется рекуррентное соотношение~\ref{eqn:dam_lev_formula}.