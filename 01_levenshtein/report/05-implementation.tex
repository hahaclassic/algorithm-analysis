\chapter{Технологический раздел}

В данном разделе будет представлена реализация алгоритмов поиска редакционного расстояния. Также будут указаны средства реализации алгоритмов и результаты тестирования.

\section{Средства реализации}

Для реализации был выбран язык программирования MicroPython~\cite{python}. Выбор обусловлен наличием функции вычисления процессорного времени в библиотеке utime~\cite{time}. Время было замерено с помощью функции time.ticks\_ms().

\section{Реализация алгоритмов}

В листинге~\ref{lst:const.py} представлено определение стоимости операций в виде глобальных переменных. Далее они будут использоваться в реализациях алгоритмов.

\includelisting
{const.py}
{Определение стоимости операций}

В листинге~\ref{lst:matrix.py} представлены функции для создания матрицы и заполнения нулевого столбца и нулевой строки расстояниями, соответствующими преобразованиям из пустой первой строки \(S_1\) в подстроки \(S_2[1...i]\), где \(i = \overline{1,m}\), и наоборот. Данные функции будут использоваться в матричных реализациях алгоритмов поиска редакционного расстояния.

\clearpage

\includelisting
{matrix.py}
{Определение функций для инициализации матрицы для матричных алгоритмов}

В листингах~\ref{lst:levenshtein_recur.py}~---~\ref{lst:damerau_dynamic.py} представлены различные реализации алгоритмов нахождения редакционного расстояния Левенштейна и Дамерау~---~Левенштейна.

\includelisting
{levenshtein_recur.py}
{Реализация рекурсивного алгоритма поиска расстояния Левенштейна}

\clearpage

\includelisting
{levenshtein_recur_cache.py}
{Реализация рекурсивного алгоритма поиска расстояния Левенштейна c мемоизацией}

\clearpage

\includelisting
{levenshtein_dynamic.py}
{Реализация матричного алгоритма поиска расстояния Левенштейна}

\includelisting
{damerau_dynamic.py}
{Реализация матричного алгоритма поиска расстояния Дамерау~---~Левенштейна}

\section{Тестирование}

В таблице~\ref{table:tests} представлены тесты для алгоритмов нахождения расстояний Левенштейна и Дамерау~---~Левенштейна. Тестирование проводилось по методологии чёрного ящика. Все тесты пройдены успешно.

\begin{table}[htb]
\caption{\centering Тесты для алгоритмов нахождения расстояний Левенштейна и Дамерау~---~Левенштейна}
\small
\centering\begin{tabular}{|c|c|c|c|c|}
      \hline
       &  &  & \multicolumn{2}{c|}{Ожидаемый результат} \\
      \cline{4-5}
      \raisebox{1.5ex}[0cm][0cm]{№} & \raisebox{1.5ex}[0cm][0cm]{$S_1$} & \raisebox{1.5ex}[0cm][0cm]{$S_2$} 
      &  Левенштейн & Дамерау~---~Левенштейн \\ \hline
      1 & $\varnothing$ & $\varnothing$ & 0 & 0 \\ \hline
      2 & $\varnothing$ & hello & 5 & 5 \\ \hline
      3 & $\varnothing$ & привет & 6 & 6 \\ \hline
      4 & beauty & $\varnothing$ & 6 & 6 \\ \hline
      5 & невероятно & $\varnothing$ & 10 & 10 \\ \hline
      6 & abc & aba & 1 & 1 \\ \hline
      7 & кошка & мышка & 2 & 2 \\ \hline
      8 & aba & aab & 2 & 1 \\ \hline
      9 & котик & котки & 2 & 1 \\ \hline
      10 & abaa & aba & 1 & 1 \\ \hline
      13 & Hello world & eHllo kord & 4 & 3 \\ \hline
    \end{tabular}
\label{table:tests}
\end{table}
