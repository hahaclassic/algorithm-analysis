\chapter{Технологический раздел}

В данном разделе будет представлена реализация алгоритмов линейного и бинарного поиска. Также будут указаны средства реализации и результаты тестирования.

\section{Средства реализации}

Для реализации был выбран язык программирования $Python$~\cite{python}. Выбор обусловлен наличием библиотеки $matplotlib$~\cite{matplotlib}. Для построения гистограмм использовалась функция $bar$~\cite{bar}. 

\section{Реализация алгоритмов}

В листингах~\ref{lst:linear.py}~---~\ref{lst:binary.py} представлены реализации алгоритмов линейного и бинарного поиска.

\includelisting
{linear.py}
{Реализация алгоритма линейного поиска}

\clearpage

\includelisting
{binary.py}
{Реализация алгоритма бинарного поиска}

\section{Тестирование}

В таблице~\ref{table:tests} представлены тесты для алгоритмов линейного и бинарного поиска. Тестирование проводилось по методологии чёрного ящика. В качестве входных данных использовались массивы из таблицы~\ref{table:arrays}. Все тесты пройдены успешно.

\begin{table}[H]
\caption{Входные массивы для тестирования алгоритмов}
\small
\centering\begin{tabular}{|c|m{10cm}|}
        \hline
        \textbf{Название массива} & \textbf{Массив} \\
        \hline
        sorted\_array & \texttt{[-1, -2, 3, 4, 5, 6, 7, 8, 9, 10]} \\
        \hline
        array & \texttt{[1, -10, 3, 4, 8, -6, 2, 7, 9, 8, -5]} \\
        \hline
        empty\_array & \texttt{[]} \\
        \hline
    \end{tabular}
\label{table:arrays}
\end{table}

\begin{table}[H]
\caption{Тесты для алгоритмов линейного и бинарного поиска}
\small
\centering\begin{tabular}{|c|c|m{5cm}|c|c|}
        \hline
        \textbf{№} & \textbf{Алгоритм} & \textbf{Описание} & \textbf{Название массива} & \textbf{Результат} \\
        \hline
        1 & Линейный поиск & Поиск значения 7 в отсортированном массиве. & sorted\_array & (6, 7) \\
        \hline
        2 & Линейный поиск & Поиск значения 7 в неотсортированном массиве. & array & (7, 8) \\
        \hline
        3 & Линейный поиск & Поиск значения 11 в неотсортированном массиве, когда значение отсутствует. & array & (-1, 11) \\
        \hline
        4 & Линейный поиск & Поиск значения 11 в пустом массиве. & empty\_array & (-1, 0) \\
        \hline
        5 & Бинарный поиск & Поиск значения 7 в отсортированном массиве. & sorted\_array & (6, 4) \\
        \hline
        6 & Бинарный поиск & Поиск значения 11 в отсортированном массиве, когда значение отсутствует. & sorted\_array & (-1, 4) \\
        \hline
        7 & Бинарный поиск & Поиск значения 11 в пустом массиве. & empty\_array & (-1, 0) \\
        \hline
    \end{tabular}
\label{table:tests}
\end{table}
