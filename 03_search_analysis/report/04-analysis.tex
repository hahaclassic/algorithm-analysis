\section{Аналитическая часть}

В данном разделе будут рассмотрены алгоритмы линейного и бинарного поиска.

\subsection{Линейный поиск}

Алгоритм, основанный на линейном поиске, или поиске полным перебором, проходит по всему массиву, пытаясь отыскать целевой элемент~\cite{stivens}. Из этого следует, что если искомое значение находится в начале массива, то оно будет найдено быстрее, нежели если оно было бы расположено в конце. Линейный поиск работает как с отсортированными, так и с неотсортированными данными.

\subsection{Бинарный поиск}

Алгоритм бинарного (двоичного) поиска позволяет осуществлять быстрый поиск в массиве $S$ отсортированных ключей. Чтобы найти ключ $q$, мы сравниваем значение $q$ со средним
ключом массива $S[n/2]$. Если значение ключа $q$ меньше, чем значение ключа $S[n/2]$,
значит, данный ключ должен находиться в верхней половине массива $S$; в противном
случае он должен находиться в его нижней половине. Повторяется данный процесс на половине, гипотетически содержащей элемент $q$, пока значение ключа $S[n/2]$ не станет равным $q$ или область поиска не станет пустой~\cite{skiena}.

