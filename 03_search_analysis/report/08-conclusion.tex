\anonsection{ЗАКЛЮЧЕНИЕ} 

В ходе лабораторной работы были проанализированы алгоритмы линейного и бинарного поиска посредством сопоставления количества сравнений для поиска заданного элемента в массиве, длина которого составляла $n = 1020$. Экспериментально было подтверждено, что линейный поиск требует $n$ сравнений в худшем случае, тогда как бинарный поиск не превышает $log_2(n)$.

Исследование также показало, что алгоритм линейного поиска может иметь меньшее количество сравнений, чем в бинарном поиске при индексах, меньших $log_2(n)$. При больших объемах данных предпочтительным остаётся бинарный поиск, но для его использования данные должны быть отсортированы.

В ходе лабораторной работы были выполнены все поставленные задачи, а именно:
    \begin{itemize}
    \item построены схемы для алгоритмов нахождения заданного значения методами линейного и бинарного поиска;
    \item создано программное обеспечение (ПО), реализующее перечисленные выше алгоритмы;
    \item проведён анализ алгоритмов по количеству необходимых сравнений для нахождения каждого элемента массива.
\end{itemize}

