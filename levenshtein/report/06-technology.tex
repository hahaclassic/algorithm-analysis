\section{Технологическая часть}

В данном разделе будет представлена реализация алгоритмов поиска редакционного расстояния. Также будут указаны средства реализации алгоритмов и результаты тестирования.

\subsection{Средства реализации}

Для реализации был выбран язык программирования MicroPython~\cite{python}. Выбор обусловлен наличием функции вычисления процессорного времени в библиотеке utime~\cite{time}. Время было замерено с помощью функции time.ticks\_ms().

\subsection{Реализация алгоритмов}

В листинге~\ref{lst:constants} представлено определение стоимости операций в виде глобальных переменных. Далее они будут использоваться в реализациях алгоритмов.
 
\begin{listing}[H]
\caption{Определение стоимости операций}
\label{lst:constants}
\begin{minted}[frame=single,fontsize = \footnotesize, linenos, xleftmargin = 1.5em]{python}
DELETE_COST = 1
INSERT_COST = 1
REPLACE_COST = 1
TRANSPOSITION_COST = 1
\end{minted}
\end{listing}

В листинге~\ref{lst:matrix} представлены функции для создания матрицы и заполнения нулевого столбца и нулевой строки расстояниями, соответствующими преобразованиям из пустой первой строки \(S_1\) в подстроки \(S_2[1...i]\), где \(i = \overline{1,m}\), и наоборот. Данные функции будут использоваться в матричных реализациях алгоритмов поиска редакционного расстояния.
 
\begin{listing}
\caption{Определение функций для инициализации матрицы для матричных алгоритмов}
\label{lst:matrix}
\begin{minted}[frame=single,fontsize = \footnotesize, linenos, xleftmargin = 1.5em]{python}
def createMatrix(rows: int, cols: int) -> list[list[int]]:
    return [[0 for _ in range(cols)] for _ in range(rows)]

def getInitialMatrix(rows: int, cols: int) -> list[list[int]]:
    matrix = createMatrix(rows+1, cols+1)
    for i in range(1, rows + 1):
        matrix[i][0] = matrix[i - 1][0] + DELETE_COST

    for j in range(1, cols + 1):
        matrix[0][j] = matrix[0][j - 1] + INSERT_COST

    return matrix
\end{minted}
\end{listing}

В листингах~\ref{lst:recur_lev}~---~\ref{lst:dyn_dam_lev} представлены различные реализации алгоритмов нахождения редакционного расстояния Левенштейна и Дамерау~---~Левенштейна.

\begin{listing}
\caption{Реализация рекурсивного алгоритма поиска расстояния Левенштейна}
\label{lst:recur_lev}
\begin{minted}[frame=single,fontsize = \footnotesize, linenos, xleftmargin = 1.5em]{python}
def RecursiveLevenshtein(s1: str, s2: str) -> int:
    length1, length2 = len(s1), len(s2)
    if length1 == 0 or length2 == 0:
        return abs(length1 - length2)
   
    if s1[-1] == s2[-1]:
        return RecursiveLevenshtein(s1[:-1], s2[:-1])
    
    return min(
        RecursiveLevenshtein(s1, s2[:-1]) + INSERT_COST,
        RecursiveLevenshtein(s1[:-1], s2) + DELETE_COST,
        RecursiveLevenshtein(s1[:-1], s2[:-1]) + REPLACE_COST)
\end{minted}
\end{listing}

\begin{listing}
\caption{Реализация рекурсивного алгоритма поиска расстояния Левенштейна c мемоизацией}
\label{lst:recur_lev_cache}
\begin{minted}[frame=single,fontsize = \footnotesize, linenos, xleftmargin = 1.5em]{python}
def RecursiveCacheLevenshtein(s1: str, s2: str, memo: dict = None) -> int:
    if memo is None:
        memo = {}

    length1, length2 = len(s1), len(s2)
    key = (length1, length2)
    if key in memo:
        return memo[key]

    if length1 == 0 or length2 == 0:
        return abs(length1 - length2)
    if s1[-1] == s2[-1]:
        distance = RecursiveCacheLevenshtein(s1[:-1], s2[:-1])
    else:
        distance = min(
            RecursiveCacheLevenshtein(s1, s2[:-1], memo) + INSERT_COST,
            RecursiveCacheLevenshtein(s1[:-1], s2, memo) + DELETE_COST,
            RecursiveCacheLevenshtein(s1[:-1], s2[:-1], memo) \
                + REPLACE_COST)
    memo[key] = distance

    return distance 
\end{minted}
\end{listing}

\begin{listing}
\caption{Реализация матричного алгоритма поиска расстояния Левенштейна}
\label{lst:dyn_lev}
\begin{minted}[frame=single,fontsize = \footnotesize, linenos, xleftmargin = 1.5em]{python}
def DynamicLevenshtein(s1: str, s2: str) -> int:
    length1, length2 = len(s1), len(s2)
    if length1 == 0 or length2 == 0:
        return abs(length1 - length2)
    
    matrix = getInitialMatrix(length1, length2)

    for i in range(1, length1 + 1):
        for j in range(1, length2 + 1): 
            cost = 0 if s1[i - 1] == s2[j - 1] else REPLACE_COST
           
            matrix[i][j] = min(
                matrix[i - 1][j] + DELETE_COST, 
                matrix[i][j - 1] + INSERT_COST,
                matrix[i - 1][j - 1] + cost)

    return matrix[length1][length2]
\end{minted}
\end{listing}

\begin{listing}
\caption{Реализация матричного алгоритма поиска расстояния Дамерау~---~Левенштейна}
\label{lst:dyn_dam_lev}
\begin{minted}[frame=single,fontsize = \footnotesize, linenos, xleftmargin = 1.5em]{python}
def DynamicDamerauLevenshtein(s1: str, s2: str) -> int:
    length1, length2 = len(s1), len(s2)
    if length1 == 0 or length2 == 0:
        return abs(length1 - length2)

    matrix = getInitialMatrix(length1, length2)

    for i in range(1, length1 + 1):
        for j in range(1, length2 + 1):
            cost = 0 if s1[i - 1] == s2[j - 1] else REPLACE_COST

            matrix[i][j] = min(
                matrix[i - 1][j] + DELETE_COST,
                matrix[i][j - 1] + INSERT_COST,
                matrix[i - 1][j - 1] + cost)

            if i > 1 and j > 1 and s1[i - 1] == s2[j - 2] \
                and s1[i - 2] == s2[j - 1]:
                matrix[i][j] = min(
                    matrix[i][j], 
                    matrix[i - 2][j - 2] + TRANSPOSITION_COST)

    return matrix[length1][length2]
\end{minted}
\end{listing}

\subsection{Тестирование}

В таблице~\ref{table:tests} представлены тесты для алгоритмов нахождения расстояний Левенштейна и Дамерау~---~Левенштейна. Тестирование проводилось по методологии чёрного ящика. Все тесты пройдены успешно.

\begin{table}[htb]
\caption{\centering Тесты для алгоритмов нахождения расстояний Левенштейна и Дамерау~---~Левенштейна}
\small
\centering\begin{tabular}{|c|c|c|c|c|}
      \hline
       &  &  & \multicolumn{2}{c|}{Ожидаемый результат} \\
      \cline{4-5}
      \raisebox{1.5ex}[0cm][0cm]{№} & \raisebox{1.5ex}[0cm][0cm]{$S_1$} & \raisebox{1.5ex}[0cm][0cm]{$S_2$} 
      &  Левенштейн & Дамерау~---~Левенштейн \\ \hline
      1 & $\varnothing$ & $\varnothing$ & 0 & 0 \\ \hline
      2 & $\varnothing$ & hello & 5 & 5 \\ \hline
      3 & $\varnothing$ & привет & 6 & 6 \\ \hline
      4 & beauty & $\varnothing$ & 6 & 6 \\ \hline
      5 & невероятно & $\varnothing$ & 10 & 10 \\ \hline
      6 & abc & aba & 1 & 1 \\ \hline
      7 & кошка & мышка & 2 & 2 \\ \hline
      8 & aba & aab & 2 & 1 \\ \hline
      9 & котик & котки & 2 & 1 \\ \hline
      10 & abaa & aba & 1 & 1 \\ \hline
      13 & Hello world & eHllo kord & 4 & 3 \\ \hline
    \end{tabular}
\label{table:tests}
\end{table}
