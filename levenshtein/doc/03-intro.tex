\anonsection{ВВЕДЕНИЕ}  %Введение

Расстояние Левенштейна (редакционное расстояние) между двумя строками --- минимальное число необходимых редакционных операций --- вставок, удалений и замен, необходимых для преобразования первой строки во вторую. При этом совпадения символов не являются операциями, которые учитываются в
редакционном расстоянии, поэтому, если строки совпадают, то их редакционное
расстояние равно нулю для любой длины строк~\cite{ulianov}. 

Расстояние Дамерау~---~Левенштейна между двумя строками --- минимальное число редакционных операций --- вставок, удалений, замен и транспозиций соседних символов, необходимых для преобразования первой строки во вторую. Таким образом, для расстояния Дамерау~---~Левенштейна кроме основных операций вставки, удаления и замены вводится дополнительная операция транспозиции (transposition) двух соседних символов.

Расстояния Левенштейна и Дамерау~---~Левенштейна используются в ряде практически важных задач: проверка правописания, задачи поиска в текстовых базах данных, задачи исследования ДНК и т.~д.~\cite{ulianov}.

\textbf{Цель лабораторной работы} --- исследование алгоритмов нахождения расстояния Левенштейна и Дамерау~---~Левенштейна. Для достижения поставленной цели необходимо выполнить следующие задачи:

\begin{itemize}
\item Построить схемы для рекурсивного алгоритма нахождения расстояния Левенштейна, рекурсивного алгоритма нахождения расстояния Левенштейна с мемоизацией, нерекурсивных алгоритмов нахождения расстояний Левенштейна и Дамерау~---~Левенштейна, основанных на идее динамического программирования;
\item создать программное обеспечение (ПО), реализующее перечисленные выше алгоритмы;
\item провести анализ потребляемых ресурсов (процессорное время и память) для перечисленных алгоритмов;
\item описать и обосновать полученные результаты в отчёте.
\end{itemize}
