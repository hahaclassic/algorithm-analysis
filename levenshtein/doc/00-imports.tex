\let\counterwithout\relax
\let\counterwithin\relax
\usepackage{float}
\usepackage{xcolor}
\usepackage{extsizes}
\usepackage{subfig}
\usepackage[export]{adjustbox}
\usepackage{tocvsec2}
\usepackage[subfigure]{tocloft}
\usepackage[newfloat]{minted}
\captionsetup[listing]{position=top}

\AtBeginEnvironment{figure}{\vspace{0.5cm}}
\AtBeginEnvironment{table}{\vspace{0.5cm}}
\AtBeginEnvironment{listing}{\vspace{0.5cm}}
\AtBeginEnvironment{algorithm}{\vspace{0.5cm}}
\AtBeginEnvironment{minted}{\vspace{-0.5cm}}

\usepackage{fancyvrb}
\usepackage{ulem,bm,mathrsfs,ifsym}
\usepackage{sectsty}

\usepackage{pdflscape}
\usepackage{geometry}
\geometry{a4paper,tmargin=2cm,bmargin=2cm,lmargin=3cm,rmargin=1cm}

\usepackage{amsthm,amsfonts,amsmath,amssymb,amscd}
\usepackage{mathtools}
\usepackage[perpage]{footmisc}

\KOMAoptions{fontsize=14pt}

\makeatletter
\def\showfontsize{\f@size{} point}
\makeatother

\makeatletter
\setlength{\@fptop}{0pt}
\makeatother

\usepackage{tempora}

\usepackage{cmap}
\usepackage[T1]{fontenc}
\usepackage[utf8]{inputenc}
\usepackage[english, main=russian]{babel}

\usepackage{indentfirst}

\usepackage{longtable}
\usepackage{multirow,makecell,array}
\usepackage{booktabs}

\usepackage{soulutf8}
\usepackage{icomma}

\usepackage{graphicx}
\usepackage{wrapfig}

\usepackage{enumitem}

\usepackage{caption}

\addto\captionsrussian{%
  \renewcommand{\listingname}{Листинг}%
}

\usepackage[figure,table,section]{totalcount}
\DeclareTotalCounter{lstlisting}
\usepackage{totcount}
\usepackage{totpages}

\linespread{1.3}

\usepackage{hyperref}

\sloppy
\clubpenalty=10000
\widowpenalty=10000

\makeatletter
\let\@@scshape=\scshape
\renewcommand{\scshape}{%
  \ifnum\strcmp{\f@series}{bx}=\z@
    \usefont{T1}{cmr}{bx}{sc}%
  \else
    \ifnum\strcmp{\f@shape}{it}=\z@
      \fontshape{scsl}\selectfont
    \else
      \@@scshape
    \fi
  \fi}
\makeatother

\usepackage{ifthen}
\newcounter{intvl}
\newcounter{otstup}
\newcounter{contnumeq}
\newcounter{contnumfig}
\newcounter{contnumtab}
\newcounter{pgnum}
\newcounter{bibliosel}
\newcounter{chapstyle}
\newcounter{headingdelim}
\newcounter{headingalign}
\newcounter{headingsize}
\newcounter{tabcap}
\newcounter{tablaba}
\newcounter{tabtita}

\setcounter{intvl}{3}
\setcounter{otstup}{0}
\setcounter{contnumeq}{1}
\setcounter{contnumfig}{1}
\setcounter{contnumtab}{1}

\setcounter{pgnum}{0}

\setcounter{bibliosel}{1}

\setcounter{chapstyle}{1}
\setcounter{headingdelim}{1}

\setcounter{headingalign}{0}

\setcounter{headingsize}{0}

\setcounter{tabcap}{0}
\setcounter{tablaba}{2}
\setcounter{tabtita}{1}
\DeclareCaptionLabelSeparator*{emdash}{~--- }
\captionsetup[figure]{labelsep=emdash,font=onehalfspacing,position=bottom,name=Рисунок}

\ifthenelse{\equal{\thetabcap}{0}}{%
    \newcommand{\tabcapalign}{\raggedright}
}

\ifthenelse{\equal{\thetablaba}{0} \AND \equal{\thetabcap}{1}}{%
    \newcommand{\tabcapalign}{\raggedright}
}

\ifthenelse{\equal{\thetablaba}{1} \AND \equal{\thetabcap}{1}}{%
    \newcommand{\tabcapalign}{\centering}
}

\ifthenelse{\equal{\thetablaba}{2} \AND \equal{\thetabcap}{1}}{%
    \newcommand{\tabcapalign}{\raggedleft}
}

\ifthenelse{\equal{\thetabtita}{0} \AND \equal{\thetabcap}{1}}{%
    \newcommand{\tabtitalign}{\raggedright}
}

\ifthenelse{\equal{\thetabtita}{1} \AND \equal{\thetabcap}{1}}{%
    \newcommand{\tabtitalign}{\centering}
}

\ifthenelse{\equal{\thetabtita}{2} \AND \equal{\thetabcap}{1}}{%
    \newcommand{\tabtitalign}{\raggedleft}
}

\DeclareCaptionFormat{tablenocaption}{\tabcapalign #1\strut}
\ifthenelse{\equal{\thetabcap}{0}}{%
    \DeclareCaptionFormat{tablecaption}{\tabcapalign #1#2#3}
    \captionsetup[table]{labelsep=emdash}
}{
    \DeclareCaptionFormat{tablecaption}{\tabcapalign #1#2\par
        \tabtitalign{#3}}
    \captionsetup[table]{labelsep=space}
}
\captionsetup[table]{format=tablecaption,singlelinecheck=off,font=onehalfspacing,position=top,skip=-5pt}
\DeclareCaptionLabelFormat{continued}{Продолжение таблицы~#2}
\setlength{\belowcaptionskip}{.2cm}
\setlength{\intextsep}{0ex}

\renewcommand{\thesubfigure}{\asbuk{subfigure}}
\captionsetup[subfigure]{font={normalsize},
    labelformat=brace,
    justification=centering,
}

\definecolor{linkcolor}{rgb}{0.0,0,0}
\definecolor{citecolor}{rgb}{0,0.0,0}
\definecolor{urlcolor}{rgb}{0,0,0}

\hypersetup{
    linktocpage=true,
    plainpages=true,
    colorlinks,
    linkcolor={linkcolor},
    citecolor={citecolor},
    urlcolor={urlcolor},
    pdflang={ru},
}
\urlstyle{same}

\setlength{\parindent}{2.5em}

\renewcommand{\labelitemi}{\bfseries~{---}}
\setlist{nosep,
    labelindent=\parindent,leftmargin=*
}

\usepackage{ragged2e}
\usepackage[explicit]{titlesec}
\usepackage{placeins}
\usepackage{xparse}
\usepackage{csquotes}

\usepackage{listingsutf8}
\usepackage{url}
\usepackage{algorithm, algorithmicx}
\usepackage[noend]{algpseudocode}
\usepackage{blkarray}
\usepackage{chngcntr}
\usepackage{tabularx}
\usepackage[backend=biber, 
    bibstyle=gost-numeric,
    citestyle=nature]{biblatex}
\newcommand*\template[1]{\text{<}#1\text{>}}
\addbibresource{biblio.bib}

\titleformat{name=\section,numberless}[block]{\normalfont\Large\centering}{}{0em}{#1}
\titleformat{\section}[block]{\normalfont\Large\bfseries\raggedright}{}{0em}{\thesection\hspace{0.25em}#1}
\titleformat{\subsection}[block]{\normalfont\Large\bfseries\raggedright}{}{0em}{\thesubsection\hspace{0.25em}#1}
\titleformat{\subsubsection}[block]{\normalfont\large\bfseries\raggedright}{}{0em}{\thesubsubsection\hspace{0.25em}#1}

\let\Algorithm\algorithm
\renewcommand\algorithm[1][]{\Algorithm[#1]\setstretch{1.5}}

\usepackage{pifont}
\usepackage{calc}
\usepackage{suffix}
\usepackage{csquotes}
\DeclareQuoteStyle{russian}
    {\guillemotleft}{\guillemotright}[0.025em]
    {\quotedblbase}{\textquotedblleft}
\ExecuteQuoteOptions{style=russian}
\newcommand{\enq}[1]{\enquote{#1}}
\newcommand{\eng}[1]{\begin{english}#1\end{english}}
\newcounter{cTheorem} 
\newcounter{cDefinition}
\newcounter{cConsequent}
\newcounter{cExample}
\newcounter{cLemma}
\newcounter{cConjecture}
\newtheorem{Theorem}{Теорема}[cTheorem]
\newtheorem{Definition}{Определение}[cDefinition]
\newtheorem{Consequent}{Следствие}[cConsequent]
\newtheorem{Example}{Пример}[cExample]
\newtheorem{Lemma}{Лемма}[cLemma]
\newtheorem{Conjecture}{Гипотеза}[cConjecture]

\renewcommand{\theTheorem}{\arabic{Theorem}}
\renewcommand{\theDefinition}{\arabic{Definition}}
\renewcommand{\theConsequent}{\arabic{Consequent}}
\renewcommand{\theExample}{\arabic{Example}}
\renewcommand{\theLemma}{\arabic{Lemma}}
\renewcommand{\theConjecture}{\arabic{Conjecture}}
%\makeatletter
\NewDocumentCommand{\Newline}{}{\text{\\}}
\newcommand{\sequence}[2]{\ensuremath \left(#1,\ \dots,\ #2\right)}

\definecolor{mygreen}{rgb}{0,0.6,0}
\definecolor{mygray}{rgb}{0.5,0.5,0.5}
\definecolor{mymauve}{rgb}{0.58,0,0.82}
\renewcommand{\listalgorithmname}{Список алгоритмов}
\floatname{algorithm}{Листинг}
\renewcommand{\lstlistingname}{Листинг}
\renewcommand{\thealgorithm}{\arabic{algorithm}}

\newcommand{\refAlgo}[1]{(листинг~\ref{#1})}
\newcommand{\refImage}[1]{(рисунок~\ref{#1})}

\renewcommand{\theenumi}{\arabic{enumi}.}
\renewcommand{\labelenumi}{\arabic{enumi}.}
\renewcommand{\theenumii}{\arabic{enumii}}
\renewcommand{\labelenumii}{(\arabic{enumii})}
\renewcommand{\theenumiii}{\roman{enumiii}}
\renewcommand{\labelenumiii}{(\roman{enumiii})}
\renewcommand{\labelitemi}{---}
\renewcommand{\labelitemii}{---}

\lstdefinelanguage{Refal}{
  alsodigit = {.,<,>},
  morekeywords = [1]{$ENTRY$},
  morekeywords = [2]{Go, Put, Get, Open, Close, Arg, Add, Sub, Mul, Div, Symb, Explode, Implode},
  %keyword4
  morekeywords = [3]{<,>},
  %keyword5
  morekeywords = [4]{e.,t.,s.},
  sensitive = true,
  morecomment = [l]{*},
  morecomment = [s]{/*}{*/},
  commentstyle = \color{mygreen},
  morestring = [b]",
  morestring = [b]',
  stringstyle = \color{purple}
}

\makeatletter
\def\p@subsection{}
\def\p@subsubsection{\thesection\,\thesubsection\,}
\makeatother
\newcommand{\prog}[1]{{\ttfamily\small#1}}
\lstset{
  backgroundcolor=\color{white},
  basicstyle=\ttfamily\footnotesize,
  breakatwhitespace=false,
  breaklines=true,
  captionpos=top,
  commentstyle=\color{mygreen},
  deletekeywords={...},
  escapeinside={\%*}{*)},
  extendedchars=true,
  inputencoding=utf8,
  frame=single,
  keepspaces=true,
  keywordstyle=\bf,
  language=Refal,
  morekeywords={<,>,$ENTRY,Go,Arg, Open, Close, e., s., t., Get, Put$},
  numbers=left,
  numbersep=5pt,
  xleftmargin=25pt,
  xrightmargin=25pt,
  numberstyle=\small\color{black},
  rulecolor=\color{black},
  showspaces=false,
  showstringspaces=false,
  showtabs=false,
  stepnumber=1,
  stringstyle=\color{mymauve},
  tabsize=8,
  title=\lstname
}
\newcommand{\anonsection}[1]{\cleardoublepage
\phantomsection
\addcontentsline{toc}{section}{\protect\numberline{}#1}
\section*{#1}\vspace*{2.5ex}
}
\newcommand{\sectionbreak}{\clearpage}
\renewcommand{\sectionfont}{\normalsize}
\renewcommand{\cftsecleader}{\cftdotfill{\cftdotsep}}
\renewcommand{\cftsecfont}{\normalfont\large}
\renewcommand{\cftsubsecfont}{\normalfont\large}
\setlength{\cftsecindent}{0pt}
\setlength{\cftsubsecindent}{0pt}
\setlength{\cftsubsubsecindent}{0pt}
\usepackage{caption}
\usepackage{amsmath}
\usepackage{amsfonts}
\usepackage{mathtools}
\DeclarePairedDelimiter\abs{\lvert}{\rvert}
\DeclarePairedDelimiter\norm{\lVert}{\rVert}
\DeclareTextCommandDefault{\textvisiblespace}{%
  \mbox{\kern.06em\vrule \@height.3ex}%
  \vbox{\hrule \@width.3em}%
  \hbox{\vrule \@height.3ex}}
\newsavebox{\spacebox}
\begin{lrbox}{\spacebox}
\verb*! !
\end{lrbox}
\newcommand{\aspace}{\usebox{\spacebox}}
\DeclareTotalCounter{listing}
\makeatletter
\renewcommand*{\p@subsubsection}{}
\makeatother
\sloppy
\def\figurename{Рисунок}
