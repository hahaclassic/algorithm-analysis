\chapter{Исследование}

\section{Технические характеристики}

Исследование проводилось на ЭВМ со следующими характеристиками:
\begin{itemize}[label=--]
    \item операционная система Ubuntu 22.04.5 LTS;
    \item объем оперативной памяти 16 ГБ;
    \item процессор Intel Core i7-8700K CPU 3.70GHz × 12~\cite{processor}.
\end{itemize}

\section{Проведение исследования}

Было проведено исследование зависимости скорости загрузки данных со страниц интернет-ресурса от количества потоков, внутри которых происходит обработка и загрузка страниц. Максимальное количество выгружаемых страниц при каждом замере равно 85. В таблице~\ref{tbl:bench} приведены результаты измерений.

\begin{table}[ht]
    \small
    \begin{center}
        \begin{threeparttable}
            \caption{Результаты измерений}
            \label{tbl:bench}
            \begin{tabular}{|r|r|}
                \hline
                \textbf{Количество потоков (шт)} & \textbf{Скорость загрузки (кол-во страниц/сек)}  \\
                \hline
                1 & 1.10 \\
                \hline
                2 & 2.21 \\
                \hline
                4 & 3.43 \\
                \hline
                8 & 5.10 \\
                \hline
                16 & 6.31 \\
                \hline
                32 & 7.64 \\
                \hline
                48 & 6.92 \\
                \hline
            \end{tabular}
        \end{threeparttable}
    \end{center}
\end{table}

\clearpage

На рисунках~\ref{img:linear}~---~\ref{img:log} продемонстрированы гистограммы, построенные на основе табличных данных.

\includeimage
{linear}
{f}
{h}
{1 \textwidth}
{Зависимость скорости загрузки страниц от количества потоков обработки (линейная шкала)}

\includeimage
{log}
{f}
{h}
{1 \textwidth}
{Зависимость скорости загрузки страниц от количества потоков обработки (логарифмическая шкала)}

Скорость загрузки страниц при использовании 48 потоков уменьшилась на 9.4\% в сравнении со скоростью при 32 потоках. Такой результат может быть связан с накладными расходами на создание, управление потоками и переключение контекста выполнения. При этом результаты измерений показали, что скорость загрузки страниц при 32 потоках в 6.9 раз выше, чем при использовании одного.
