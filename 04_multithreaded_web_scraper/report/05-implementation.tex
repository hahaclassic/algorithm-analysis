\chapter{Пример работы программы}

На листинге~\ref{lst:start.txt} изображён пример запуска программы в терминале. При запуске в аргументах командой строки указываются входные параметры. Все аргументы являются опциональными, т.~е. если какой-либо аргумент опущен, то его значение будет стандартным (значением по умолчанию) для данного параметра. Список значений по умолчанию:

\begin{itemize}[label=--]
    \item 1 для workers (максимальное количество потоков);
    \item 10 для pages (максимальное количество страниц);
    \item \url{https://edimdoma.ru} для url (адрес главной страницы ресурса);
    \item ../data для dir (путь до директории хранения выгруженных файлов).
\end{itemize}

\includelisting
{start.txt}
{Пример запуска программы}

На рисунке~\ref{img:source} показан фрагмент реальной страницы ресурса, а на рисунке~\ref{img:downloaded} фрагмент загруженного HTML файла. Так как внутри HTML страницы интернет-ресурса \url{https://edimdoma.ru} располагались элементы каскадной таблицы стилей (CSS), то многие элементы внешнего оформления страницы остались неизменными.

\includeimage
{source}
{f}
{h}
{1 \textwidth}
{Фрагмент страницы интернет-ресурса}

\includeimage
{downloaded}
{f}
{h}
{1 \textwidth}
{Фрагмент загруженной HTML страницы}

\clearpage

\chapter{Тестирование}

В таблице~\ref{tbl:test} представлены результаты функционального тестирования программы. Все тесты пройдены успешно.

\begin{table}[ht]
    \small
    \begin{center}
        \begin{threeparttable}
            \caption{Результаты функционального тестирования программы}
            \label{tbl:test}
            \begin{tabular}{|p{3cm}|p{3cm}|p{4cm}|p{4cm}|}
                \hline
                \textbf{Количество страниц} & \textbf{Количество потоков} & \textbf{Ожидаемый результат (кол-во загруженных страниц)} & \textbf{Полученный результат (кол-во загруженных страниц)}  \\
                \hline
                1 & 1  & 1 & 1 \\
                \hline
                1  & 12 & 1 & 1  \\
                \hline
                1  & 32 & 1 & 1  \\
                \hline
                100  & 1 & 100 & 100 \\
                \hline
                100  & 12 & 100 & 100 \\
                \hline
                100  & 32 & 100 & 100 \\
                \hline
                500  & 1  & 500 & 500 \\
                \hline
                500  & 12 & 500 & 500 \\
                \hline
                500  & 32 & 500 & 500 \\
                \hline
            \end{tabular}
        \end{threeparttable}
    \end{center}
\end{table}
