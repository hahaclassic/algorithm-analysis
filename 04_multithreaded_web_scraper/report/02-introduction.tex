\chapter*{ВВЕДЕНИЕ}
\addcontentsline{toc}{chapter}{ВВЕДЕНИЕ}

При использовании нескольких потоков управления можно спроектировать приложение, которое будет решать одновременно несколько задач в рамках единственного процесса, где каждый поток решает отдельную задачу~\cite{unix}. Одно из преимуществ такого подхода заключается в том, что решение некоторых задач можно разбить на более мелкие подзадачи, что может дать прирост производительности программы~\cite{unix}. 

Однопоточный процесс, выполняющий решение нескольких задач, неявно вынужден решать их последовательно, поскольку имеет только один поток управления. При наличии нескольких потоков управления независимые друг от друга задачи могут решаться одновременно отдельными потоками. Две задачи могут решаться одновременно только при условии, что они не зависят друг от друга~\cite{unix}.

\textbf{Цель работы} -- Получить навык организации параллельных вычислений на основе нативных потоков. 

Для достижения цели необходимо выполнить следующие задачи:
\begin{itemize}[label=--]
    \item описать входные и выходные данные программы;
    \item разработать программное обеспечение (ПО), которое осуществляет выгрузку данных со страниц указанного интернет-ресурса;
    \item провести исследование скорости скачивания страниц в зависимости от количества созданных потоков. 
\end{itemize}
